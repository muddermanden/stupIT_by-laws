\documentclass[11pt,a4paper]{article}
\title{stupIT bylaws}
\date{April 28, 2014}
\author{General Assembly of the 28th of April 2014}
\newcommand{\HRule}{\noindent\rule{\linewidth}{0.5mm}}

\makeatletter
% Format the paragraphs. The command expects an argument (the number of the paragraph).
\renewcommand{\paragraph}[1]{\@startsection{paragraph}{4}{\z@}%
    {-3.25ex\@plus -1ex \@minus -.2ex}%
    {0ex}
{\normalfont\normalsize\bfseries}{}{\bf{\S~#1.}}}
\renewcommand{\subparagraph}[1]{\@startsection{subparagraph}{5}{\z@}%
    {-3.25ex\@plus -1ex \@minus -.2ex}%
    {0ex}
{\normalfont\normalsize\bfseries}{}{\bf{pt.~#1.}}} 

\makeatother

\begin{document}
\maketitle

\HRule{}
\section*{Name and address}
\paragraph{1} The name of the organisation is \emph{The Student Union at the IT University of Copenhagen}, abbreviated \emph{stup.IT}
\subparagraph{1}The organisation is situated at the \emph{IT University of Copenhagen (ITU); Rued Langgaards Vej 7, 2300, K\o benhavn S, att: Student Union}
\subparagraph{2} The organisation is a non-profit organisation (Almindelig forening).

\section*{Purpose}
\paragraph{2} The student union aims to improve the condition of students enrolled at ITU.
\subparagraph{1} The purpose of the Student Union is to defend and promote the student's interests and represent them through a united voice.

\section*{Organisational structure}
\paragraph{3} The Student Council consists of the standing members.
\subparagraph{1} The standing members consists of the board of stup.IT and the student representatives of stup.IT in the collegiate bodies and official organisational groups at ITU.
\subparagraph{2} The Board consist of the presidium and four ordinary members. They are all elected at the general assembly.
\subparagraph{3} The Board has the responsibility to keep a current list of standing members of the Student Council and make this list available to the Student Council and ITU students in general.
\section*{The Student Council}
\paragraph{4} The Student Council is responsible for the work of the organisation, and is the highest ranking political and organisational authority between the general assemblies.
\subparagraph{1} The Student Council is also responsible for the work of established committees within the Student Council.
\paragraph{5} The members of the Student Council each have one vote in the Student Council. External participants can be granted right of speech by the council.
\paragraph{6} The Student Council convenes at least four times per semester. The meetings should be evenly spread out over the semester.
\subparagraph{1} The Student Council meetings must be open.
\subparagraph{2} All students at ITU have the right of speech at the Student Council meetings.
\subparagraph{3} Each meeting must be convoked at least one week in advance.
\subparagraph{4} If three or more persons from the council requests a meeting to the presidium, a council
meeting must be held at least two weeks from that date.
\subparagraph{5} The Student Council elects, from its midst, people in charge of moderating, taking minutes and counting votes at each meeting.

\section*{The Board}
\paragraph{7} The Board administers the daily work of the Student Union and acts on a mandate from the general assembly and the council.
\subparagraph{1} The Board can make decisions on matters that cannot wait until the next meeting of the Student Council.
\subparagraph{2} Decisions made in accordance with \S 7 pt. 2 must be presented to the council as soon as possible.
\subparagraph{3} The Board shall inform the Student Council of all major financial decisions, if possible before they are taken or at least at the next Student Council meeting.
\subparagraph{4}. The Board shall provide the Student Council with a financial report when asked by the Student Council.
\subparagraph{5} The stup.IT board can invite external partners to be part of an Advisory Panel on matters related to the well-being of the organisation, funding and other strategic advice.
\paragraph{8} The members of the Board each have one vote. External participants can be granted right of speech by the Board.
\paragraph{9} If the president is absent over a period of time the vice-president takes his or her place until the coming general assembly.
\subparagraph{1} If a member of the Board is absent over a period of time the council elects a member from its midst to replace him or her in the Board.

\section*{The presidium}
\paragraph{10} The presidium consists of a President, Vice-President and Treasurer.
\subparagraph{1} The presidium holds the responsibility for the daily organisational, financial and political administration.
\subparagraph{2} The presidium is responsible for convoking council and Board meetings.
\subparagraph{3} Compensation can be provided for one or more presidium members in the framework defined by Student Council or General Assembly.
The framework must be approved by 2/3 majority

\section*{Student elected representatives}
\paragraph{11} At university elections the Student Union runs on a common electoral roll.
\subparagraph{1} The candidates on the electoral roll of the Student Union are elected by the Student
Council.
\paragraph{12} The student elected representatives on the electoral roll are expected to keep the students and the Student Union informed about their work and relevant bullets on agendas, as well as participate in the work of the Student Council.
\subparagraph{1} If a representative do not follow the decisions of the council the council can:
\begin{itemize}
    \item[A. ]{Reprimand the representative.}
    \item[B. ]{Announce that the representative are no longer welcome in the council.}
    \item[C. ]{Prevent the representative from being reelected on the election roll of the Student Council.}
\end{itemize}

\section*{Student Activity Fund}
\paragraph{13} stup.IT board manages the Student Activity Fund.
\subparagraph{1} stup.IT supports student activities in accordance to the Student Activity Fund Guidelines.

\section*{General Assembly}
\paragraph{14} The General Assembly is the highest authority of the Student Union.
\subparagraph{1} The Board must convoke a General Assembly no later than 2 weeks before the actual meeting.
\subparagraph{2} Amendments to the bylaws must be received by the Board, 1 week before the meeting at the latest.
\subparagraph{3} Amendments to the bylaws should be published 6 days before the meeting at the latest.
\subparagraph{4} The budget and financial report needs to be public 2 weeks before the actual meeting. Amendments to the budget and financial report, must be received by the Board 1 week before the actual meeting.
\subparagraph{5} An ordinary General Assembly is held annually.
\subparagraph{6} The agenda for the annual General Assembly shall at least include the following items:
\subparagraph{7} Amendments to amendments to the bylaw\footnote{Amendments to amendments to the bylaw should be published 2 days before the meeting at the latest.} must be received 3 days before the general assembly at the latest.
\subparagraph{8} Amendments to amendments to the bylaw should be published 2 days before the meeting at the latest.
\begin{enumerate}
    \item{Election of a chairman, secretary and two vote counters for the general assembly.}
    \item{Annual report from the president.}
    \item{Approval of the annual report.}
    \item{Debate and vote on proposed changes for by-laws and work-programme.}
    \item{Presentation of the budget by the treasurer.}
    \item{Debate and vote on proposed changes for the budget.}
    \item{Election of President, Vice-President and Treasurer.}
    \item{Election of 4 Board members.}
    \item{Election of 2 auditors.}
    \item{Any other business.}
\end{enumerate}
\subparagraph{6} The elected auditors, Board members, President, Vice President and Treasurer enters into office a month before the official beginning of the following semester. The previous Board should during that time meet with the newly elected Board regularly, to hand over any work and experience that might be relevant for the new Board.

\paragraph{15} Proposals for the work programme, shall be received by the Board, no later than one week before the assembly.
\subparagraph{1} Candidates for the Presidium and Board must be submitted in writing to the Board no later than a week before the assembly.
\subparagraph{2} To run for the Board the candidates must be enrolled at ITU at the time of the general assembly.
\paragraph{16} All students at the IT University have one vote. The vote can be retrieved by presenting a valid ITU student card to an authority elected by the presidium.
\subparagraph{1} The General Assembly makes its decisions by simple majority (one vote more than half of the valid votes) and by show of hands.
\subparagraph{2} At an election of one or more individuals written voting is applied. Also, the rule of simple majority applies.
\paragraph{17} All students have the right of speech at the general assembly. External participants can be granted a right to speak by the general assembly.

\section*{Extraordinary General Assembly}
\paragraph{18} Extraordinary General Assembly may be held whenever the council deems it necessary, and must be held when at least 1/3 of the council members have made a written request to the President. In such cases the extraordinary general assembly must be held no later than four weeks after the request has been made.
\subparagraph{1} The deadline for convoking an extraordinary general assembly is 2 weeks.
\section*{Financials, accounting and auditing}
\paragraph{19} The organisation's fiscal year runs from 1st of January to 31st of December.
\subparagraph{1} The Board, under accountability to the general assembly, should strive to follow the budget.
\subparagraph{2} The council's records are kept by the Treasurer.
\subparagraph{3} The accounts are audited by the auditors appointed by the General Assembly. The auditors must be provided with a full account of transactions two weeks before the General Assembly.
\subparagraph{4} Auditors must not be members of the current serving Student Council.
\paragraph{20} The organisation is legally bound by the signatures of the President and at least one Board member who are not a member of the presidium.
\subparagraph{1} To make a loan for the union, and/or to sell or pawn the organisation's properties, the signatures of all the members of the Board are required.

\section*{Amendments to the by-laws}
\paragraph{21} Amendments can be amended with simple majority at a general assembly.
\subparagraph{1} \S 1, \S 2 and \S 3 can only be amended by $2/3$ majority.
\subparagraph{2} Amendments take effect after they are passed.

\section*{Dissolution}
\paragraph{22} Dissolution of the union can only take place with $2/3$ majority at two consecutive general assemblies, at least one month apart. One of the general assemblies must be ordinary.
\subparagraph{1} The union's assets, in the event of dissolution, are to be used in accordance with the organisation's purposes (see \S 2) or other charitable causes. Resolution on the specific use of the property are taken by the disintegrating general assembly.

\end{document}
