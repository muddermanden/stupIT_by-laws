\documentclass[11pt,a4paper]{article}
\title{stupIT Student Council: Rules of procedure}
\date{February 10, 2014}
\author{Proposal Awaiting Enactment}
\newcommand{\HRule}{\noindent\rule{\linewidth}{0.5mm}}


\makeatletter
% Format the paragraphs. The command expects an argument (the number of the paragraph).
\renewcommand{\paragraph}[1]{\@startsection{paragraph}{4}{\z@}%
  {-3.25ex\@plus -1ex \@minus -.2ex}%
  %{1.5ex \@plus .2ex}%
  {0ex}
  {\normalfont\normalsize\bfseries}{}{\bf{\S\space#1.}}}
% Format the sub-sections. The command expects the number of the sub-section as an argument.
\renewcommand{\subparagraph}[1]{\@startsection{subparagraph}{5}{\z@}%
  {-3.25ex\@plus -1ex \@minus -.2ex}%
  %{1.5ex \@plus .2ex}%
  {0ex}
  {\normalfont\normalsize\bfseries}{}{\bf{pt.\space#1.}}} 

\makeatother

\begin{document}
\maketitle

\HRule
\section*{Constitution}
\paragraph{1} The student council has been constituted as per \S 2 pt. 1 in the bylaws of stupIT.

\paragraph{2} The job of the student council is to discuss and decide upon matters relevant to students at the IT-University of Copenhagen in the absence of the general assembly, and to work towards the goal(s) described in the purpose section in the bylaws of stupIT.

\paragraph{3} On the first council meeting after a general assembly, the council reviews and agrees upon the rules and procedure and elects speakers (see \S 11).
\subparagraph{2} These rules and procedures can be changed before any meeting by a simple majority, it these changes have been distributed to the council members by the procedures described under convocation.

\section*{Convocation}
\paragraph{4} According the bylaws of stupIT the student council convenes at least four times per semester.

\paragraph{5} At the beginning of each semester the board of stupIT presents dates for future council meetings.

\paragraph{6} A student council meeting must be convoked by the speaker as soon as possible, if 30\% or more of the members of the student council requests it.

\paragraph{7} Council meetings are public.

\section*{Voting (quorum)}
\paragraph{8} Unless otherwise stated, decisions made by the council are passed with 50\% or more in favour.

\paragraph{9} The council can only make decisions when at least 6 members are present.

\section*{Agendas}
\paragraph{10} The agenda are to be published at least five days before the actual meeting.
\subparagraph{2} Proposals for the agenda are to be received by the speakers at least a week before the meeting.

\paragraph{11} The agenda must at least include the following:
\begin{itemize}
    \item[A. ]{Approval of of minutes from previous meeting.}
    \item[B. ]{Approval of agenda.}
\end{itemize}

\section*{Speakers}

\paragraph{12} The speaker group consists of a speaker, vice-speaker and secretary.
\subparagraph{2} The speakers are elected by the student council.
\subparagraph{3} The speaker, vice-speaker or secretary cannot be a student elected representative (SER).
\subparagraph{4} The speakers are to be neutral in political matters concerning the student council and cannot take part in the debate in meetings.

\paragraph{13} The speaker, vice-speaker and secretary are elected by the council by a simple majority. 

\paragraph{14} The speaker is responsible for conducting the student council meeting, approving agendas, convoking meetings and the daily operation of the council outside council meetings.

\subparagraph{2} When the speaker is absent the vice-speakers steps in with similar responsibilities as the speaker.

\paragraph{15} The secretary is responsible for writing down important and relevant information from the council meetings.

\subparagraph{2} The secretary is responsible for internally publishing the minutes taken at council meetings within two days of the meeting.

\subparagraph{3} After four days of the meeting the notes are to be published to the public via social media.

\subparagraph{4} Changes to the minutes from student council members must be handed in before it is publicised.

\section*{Mistrust}
\paragraph{16} At any time during a council meeting, members of the council can call for a mistrust voting against the speakers.

\subparagraph{2} A vote for mistrust passes with 30\% or more in favour.

\paragraph{17}
If a mistrust vote is passed at a council meeting, the council must at that same meeting elect new speakers with the same responsibilities as the previous speakers.

\end{document}
